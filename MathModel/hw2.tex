\documentclass [12pt] {article}

% report setting environment
\usepackage {fancyhdr}
\pagestyle {fancy}
\fancyhf{}
\fancyheadoffset {0in}
\fancyhead[C] {HW2-R05229011}
\fancyhead [LO] {\thepage}

\begin {document}
	\section {Struggle for Existence from Charles Darwin}
	\subsection {short summary}
	In this article, Darwin thought that a species won't growth unlimitedly. The population of a species will be constraineded by many factors, such as food (grass or other animal) or living resources (competition from same species or different species). Thus, any species will have its proper amount of population.
	\subsection {some equations}
	From
	\begin {itemize}
		\item the high rate at which all organic beings tend to increase
		\item as more individuals are produced than can possibly survive, there must in every case be a struggle for existence, either one individual with another of the same species, or with the individuals of distinct species, or with physical conditions of life
	\end {itemize}
	For two species $x, y$, we have
	\begin {equation}
	\frac{dx}{dt} = \alpha_{1}x - \beta_{1}x^{2} - \gamma_{2}xy - \delta_{1}
	\end {equation}
	\begin {equation}
	\frac{dy}{dt} = \alpha_{2}y - \beta_{2}y^{2} - \gamma_{1}xy - \delta_{2}
	\end {equation}
	with $\alpha$ is the growth rate for species, $\beta$ is the competition in same species, $\gamma$ is the competition between distinct species, and $\delta$ is the physical conditions of life.
	\section {Radiocarbon Dating}
	(1)
	\[\frac{dN}{dt} = -kN \Rightarrow \frac{dN}{N} = -k dt\]
	\[\Rightarrow \int_{t_{0}}^{t}\frac{dN}{N} = \int_{t_0}^{t} k dt \Rightarrow \ln(\frac{N(t)}{N(t_0)}) = -k(t-t_0)\]
	\[Remove\; log\; \Rightarrow N(t) = N(t_0)e^{-k(t-t_0)}\]
	\vspace {16pt}
	\\(2)\\
	I use the equation : $\ln p = -k (t_1 - t_0)$\\
	The year for dated : 1994 = $t_1$\\
	\begin {itemize}
		\item case1: 75\% of the initial level of carbon 14
		\[t_0 = 1994 + \frac{\ln 0.75}{0.0001216} \approx -371\]
	The year is around 371 B.C.
		\item case2: 77\% of the initial level of carbon 14
	\[t_0 = 1994 + \frac{\ln 0.77}{0.0001216} \approx -155\]
	The year is around 155 B.C.
	\end {itemize}
	Thus, the scroll is written between 371 B.C. to 155 B.C.\\
	\vspace {16pt}
	\\(3)\\
	Problem1: what is the proportion of the original from 500 A.C. to this time (2017 A.C.). Apply the equation $\ln p = -k (t_1 - t_0)$, We have\\
	\[p = e^{-0.0001216 * (2017-500)} \approx 0.832\]
	Thus, the proportion of the original carbon 14 is 83.2\%\\
	Problem2:\\
	The round table is dated in 1976 and with 91.6\% of original quantity. Apply the equation $\ln p = -k (t_1 - t_0)$, we can get
	\[t_0 = 1976 + \frac{\ln 0.916}{0.0001216} \approx 1254\]
	Thus, 1254 is the table date.
	\vspace {16pt}
	\\(4)\\
	If we find that the 85.5 \% of the C-14, what's the age?\\
	By applying the equation $\ln p = -k (t_1 - t_0)$, we can get
	\[age = t_1 - t_0 = \frac{\ln 0.855}{-0.0001216} \approx 1288\]
	Thus, the age is 1288 years.
	\vspace {16pt}
	\\(5)\\
	Assume the half life of C-14 is 5700 years.\\
	Since the radiocative atoms decay is exponential decay, after certain years, the change of proportion between two years are not significant. For example, the proportion between age 10 billion and 11 billion are nearly zero. So the time range for C-14 should not be too long. If we need to use long years, we need choose another atom to use.
	\vspace {16pt}
	\\(6)\\
	The half-time is 3.3 hr, so k is
	\[k = \frac {\ln 2}{3.3} \approx 0.21 \;\;[hr^{-1}]\]
	By applying the equation $\ln p = -k (t_1 - t_0)$, we can get
	\[hours = t_{1}-t_{0} = \frac{\ln 0.9}{-0.21} \approx 0.5\]
	Thus, it take 0.5 hr for 90\% of the lead to decay.
\end {document}
