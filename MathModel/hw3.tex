\documentclass [12pt] {article}

% report setting environment
\usepackage {fancyhdr}
\pagestyle {fancy}
\fancyhf{}
\fancyheadoffset {0in}
\fancyhead[C] {HW3-R05229011}
\fancyhead [LO] {\thepage}

\begin {document}
	\section {Problem 1}
	\subsection {Q1 - solve $\overline{x}$ and $\overline{y}$}
	Since x, y $>$ 0, $\overline{x}$ is one of point in x and $\overline{y}$ is one of point in y, we have $\overline{x}$, $\overline{y}$ $>$ 0. So we have
	\[f(\overline{x}, \overline{y}) = \alpha_1\overline{x}(1-\overline{x}-a\overline{y})=0=1-\overline{x}-a\overline{y}\]
	\[g(\overline{x}, \overline{y}) = \alpha_2\overline{y}(1-\overline{y}-b\overline{x})=0=1-\overline{y}-b\overline{x}\]
	By solving above two equations, we have
	\[\overline{x} = \frac{a-1}{ab-1} \;and \;\overline{y} = \frac{b-1}{ab-1}, \;with \;ab\neq 0\]
	By the restriction of $\overline{x}$, $\overline{y}$ $>$ 0, $\overline{x}$ and $\overline{y}$ exist only under following conditions:
	\begin {itemize}
		\item $a-1 \neq 0$
		\item $b-1 \neq 0$
		\item $ab-1 \neq 0$
		\item $(ab-1)(a-1) > 0$
		\item $(ab-1)(b-1) > 0$
	\end {itemize}
	\subsection {Q2 - what is A, tr(A) and det(A)}
	Given f and g, we have
	\[\frac{\partial f}{\partial x} = \alpha_1 - 2\alpha_1x-\alpha_1ay\]
	\[\frac{\partial f}{\partial y} = -\alpha_1ax\]
	\[\frac{\partial g}{\partial x} = -\alpha_2by\]
	\[\frac{\partial g}{\partial y} = \alpha_2 - 2\alpha_2y-\alpha_2bx\]
	So,
	\[A = \left( \begin{array}{cc} \alpha_1-2\alpha_1x-\alpha_1ay & -\alpha_1ax \\ -\alpha_2by & \alpha_2-2\alpha_2y-\alpha_2bx\end{array} \right)\]
	\[tr(A) = \frac{\partial f}{\partial x} + \frac{\partial g}{\partial y} = (\alpha_1+\alpha_2) - 2(\alpha_1x+\alpha_2y)-(\alpha_1ay+\alpha_2bx)\]
	\[det(A) = \frac{\partial f}{\partial x}\frac{\partial g}{\partial y} - \frac{\partial f}{\partial y}\frac{\partial g}{\partial x} = \]
	\[\alpha_1\alpha_2 - 2\alpha_1\alpha_2 x - 2\alpha_1\alpha_2 y + 4\alpha_1\alpha_2 xy + 2a\alpha_1\alpha_1 y^2 + 2b\alpha_1\alpha_2 x^2 - a\alpha_1\alpha_2 y - \alpha_1\alpha_2 bx\]
	\subsection {Q3 - prove that $(\overline{x}, \overline{y})$ is stable equilibrium point}
	Assume $\lambda_1$ and $\lambda_2$ are two solutions for $\lambda^2 - tr(A)\lambda+det(A) = 0$\\
	With $tr(A) < 0$ and $det(A) > 0$, we have $\lambda_1+\lambda_2 < 0$ and $\lambda_1\lambda_2 > 0$. Thus, $\lambda = -c \pm di$ with $c, d \in \Re$ and $c > 0.$\\
	By the general solution $\widetilde{u} = e^{\lambda t}\widetilde{c} = e^{(-c\pm di)t}\widetilde{c}$. We have
	\[e^{(-c\pm di)t}\widetilde{c} = e^{-ct}(e^{\pm dit}\widetilde{c}) \rightarrow 0 \;as \;t \rightarrow \infty \;causes \;e^{-ct}\rightarrow 0\]
	Thus, $(\overline{x}, \overline{y})$ is a stable equilibrium point.
	\subsection {Q4}
	If (a, b) = (0.5, 0.3), we have $\overline{x} = \frac{10}{17}$ and $\overline{y} = \frac{14}{17}$\\
	By phase diagram, we have ... TODO\\
	According to Q3, tr(A) = $\frac{0.5\alpha_1+0.7\alpha_2}{-0.85} < 0$ and det(A) = $\frac{0.35\alpha_1\alpha_2}{0.85} > 0$ (since $\alpha_1$ and $\alpha_2$ are $>$ 0), the point $(\overline{x}, \overline{y})$ is a stable equilibrium poing.

	\section {Problem 2}
	\subsection {Q1}
	Original equation:
	\[Gaussian = \frac{1}{\sqrt{2\sigma^2\pi}}e^{\frac{-(x-x_o)^2}{2\sigma^2}}\]
	Modified equation:
	\[Modified \;Gaussian = \frac{1}{\sqrt{2\sigma^2\pi}}e^{\frac{-(x-x_o)^2}{2\sigma^2}}\]
	\subsection {Q2}
	If using Delta func, we may set
	\[Delta \;func = \left\{ \begin{array}{rcl} 0.2 & \mbox{for} & 12.5\leq x\leq17.5 \\
												0 & \mbox{for} & otherwise\end{array} \right.\]
	But the probability for each element in (12.5, 17.5) are equal.
\end {document}
